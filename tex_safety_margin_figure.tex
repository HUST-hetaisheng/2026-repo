% ============================================================
% Safety Margin Correlation Figure - LaTeX Text
% 安全边际对比图 - 论文文本
% ============================================================

% ------------------------------------------------------------
% Figure Environment
% ------------------------------------------------------------
\begin{figure}[htbp]
    \centering
    \includegraphics[width=0.85\textwidth]{figures/safety_margin_correlation_beautiful.png}
    \caption{Safety margin correlation between fan vote share and rule preference. 
    The $y$-axis represents the rank difference (Rank Rule Rank $-$ Percentage Rule Rank), 
    where positive values indicate that the contestant achieves a better (lower) rank under the Percentage Rule. 
    The regression line has a slope of $2.40$ ($p < 10^{-15}$, $n = 2777$), 
    providing strong statistical evidence that higher fan support leads to improved rankings 
    under the Percentage Rule compared to the Rank Rule.}
    \label{fig:safety_margin}
\end{figure}

% ------------------------------------------------------------
% Main Text Description (可直接插入正文)
% ------------------------------------------------------------

To quantify the differential impact of the two aggregation methods on contestants with varying levels of fan support, we computed the \textit{rank advantage} metric for each contestant-week observation:
\begin{equation}
    \Delta_{\text{rank}} = R_{\text{Rank}} - R_{\text{Pct}}
    \label{eq:rank_advantage}
\end{equation}
where $R_{\text{Rank}}$ denotes the contestant's placement under the Rank Rule and $R_{\text{Pct}}$ denotes the placement under the Percentage Rule. A positive $\Delta_{\text{rank}}$ indicates that the contestant ranks higher (i.e., is safer from elimination) under the Percentage Rule.

Figure~\ref{fig:safety_margin} presents the scatter plot of $\Delta_{\text{rank}}$ against fan vote share for all $n = 2777$ contestant-week observations. The regression analysis reveals a statistically significant positive relationship:
\begin{equation}
    \Delta_{\text{rank}} = 2.40 \times \text{Fan Vote Share} - 0.30
    \label{eq:regression}
\end{equation}
with $R^2 = 0.023$ and $p < 10^{-15}$. 

The positive slope of $\beta = 2.40$ provides mathematical proof of the \textbf{populist bias} inherent in the Percentage Rule: for every 10 percentage point increase in fan vote share, a contestant's rank improves by approximately 0.24 positions under the Percentage Rule relative to the Rank Rule. This effect, while modest in magnitude, is highly significant and has substantial implications for elimination outcomes.

The figure clearly delineates two zones:
\begin{itemize}
    \item \textbf{Percentage Rule Advantage Zone} ($\Delta_{\text{rank}} > 0$, green region): Contestants in this region benefit from the Percentage Rule, typically those with high fan support but potentially lower judge scores.
    \item \textbf{Rank Rule Advantage Zone} ($\Delta_{\text{rank}} < 0$, red region): Contestants here are better protected by the Rank Rule, typically those with strong technical performance (high judge scores) but lower popular appeal.
\end{itemize}

This analysis demonstrates that the choice of aggregation method is not value-neutral. The Percentage Rule systematically favors contestants with strong fan bases, effectively amplifying the voice of the audience in determining competition outcomes. Conversely, the Rank Rule provides a more balanced framework that gives greater weight to professional evaluation, protecting technically skilled contestants who may lack widespread popularity.

% ------------------------------------------------------------
% Shorter Version (简短版本,适合空间有限时)
% ------------------------------------------------------------

% Figure~\ref{fig:safety_margin} illustrates the relationship between fan vote share and the rank advantage under the Percentage Rule. The positive regression slope ($\beta = 2.40$, $p < 10^{-15}$) confirms that contestants with higher fan support systematically achieve better rankings under the Percentage Rule compared to the Rank Rule, providing quantitative evidence of the populist bias in percentage-based aggregation.

% ------------------------------------------------------------
% Key Statistics Table (可选)
% ------------------------------------------------------------

\begin{table}[htbp]
\centering
\caption{Regression Analysis: Fan Vote Share vs. Rank Advantage}
\label{tab:regression_stats}
\begin{tabular}{lc}
\toprule
\textbf{Statistic} & \textbf{Value} \\
\midrule
Sample size ($n$) & 2,777 \\
Slope ($\beta$) & 2.40 \\
Intercept ($\alpha$) & $-0.30$ \\
$R^2$ & 0.023 \\
$p$-value & $< 10^{-15}$ \\
Standard error & 0.29 \\
\bottomrule
\end{tabular}
\end{table}

% ------------------------------------------------------------
% Interpretation Paragraph (解释段落)
% ------------------------------------------------------------

The low $R^2$ value (0.023) indicates that fan vote share alone explains only a small fraction of the variance in rank advantage, which is expected given the multitude of factors influencing competition outcomes. However, the extremely low $p$-value ($< 10^{-15}$) confirms that this relationship is not due to chance. The practical significance lies in the direction and magnitude of the effect: the Percentage Rule consistently provides a slight but systematic advantage to fan-favorite contestants, which accumulates over multiple weeks and can determine the difference between survival and elimination in close competitions.

This finding aligns with our earlier observation that in the 93 weeks where the two methods disagreed on elimination outcomes, the Percentage Rule favored the contestant with higher fan support in 56 cases (60.2\%), compared to only 37 cases (39.8\%) for the Rank Rule—a ratio of 1.51:1.
