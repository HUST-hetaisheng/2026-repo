%%
%% MCM 2026 Problem C - DWTS Fan Vote Estimation
%%
\documentclass{mcmthesis}
\mcmsetup{tstyle=\color{red}\bfseries,
        tcn = 2607256, problem = C,
        sheet = true, titleinsheet = false, keywordsinsheet = true,
        titlepage = false, abstract = true}

\usepackage{newtxtext,newtxmath}
\usepackage{indentfirst}
\usepackage{algorithm}
\usepackage{algpseudocode}
\usepackage{booktabs}

\title{Inverse Estimation of Fan Votes in Dancing with the Stars}
\date{\today}

\begin{document}
\begin{abstract}
We address the inverse problem of estimating fan votes from DWTS elimination data across 34 seasons (2005--2024). The challenge: given only judges' scores and elimination outcomes, infer the hidden fan vote distribution. We develop regime-specific models for three distinct voting rules: Rank-based (S1--2), Percent-based (S3--27), and Bottom2+Judges' Save (S28--34). Our approach uses convex optimization with softmax likelihood and entropy regularization, achieving 100\% constraint satisfaction for Percent seasons and identifying ``upset'' eliminations in Rank seasons where strong fan opposition overcame judges' preferences. Certainty is quantified via ensemble perturbation, with Percent regime showing highest certainty (0.975), followed by Rank (0.913) and Bottom2 (0.821).

\begin{keywords}
inverse problem; fan vote estimation; convex optimization; DWTS; certainty quantification
\end{keywords}
\end{abstract}
\maketitle
\tableofcontents
\newpage

%% ===========================================
\section{Introduction}
%% ===========================================

\subsection{Problem Overview}
Dancing with the Stars (DWTS) combines professional judges' scores with fan votes to determine eliminations. Fan vote totals are never disclosed, creating an \textbf{inverse problem}: given elimination outcomes, estimate the underlying fan votes.

We address four questions:
\begin{enumerate}
    \item Estimate fan vote totals for all contestants across 34 seasons.
    \item Quantify \emph{certainty} in these estimates.
    \item Analyze whether certainty varies by contestant/week.
    \item Evaluate rule fairness and recommend improvements.
\end{enumerate}

\subsection{Three Voting Regimes}
DWTS has used three distinct scoring rules:
\begin{itemize}
    \item \textbf{Rank} (S1--2): Combined rank = judge rank + fan rank. Highest sum eliminated.
    \item \textbf{Percent} (S3--27): Combined score = judge share + fan share. Lowest sum eliminated.
    \item \textbf{Bottom2} (S28--34): Bottom two by combined rank; judges then choose whom to save.
\end{itemize}

\begin{figure}[h]
\centering
\includegraphics[width=0.75\textwidth]{figures/rules_of_score_combining_and_couple_elimination.png}
\caption{Scoring rule evolution across seasons}
\label{fig:rules}
\end{figure}

%% ===========================================
\section{Model: Inverse Estimation via Convex Optimization}
%% ===========================================

\subsection{Problem Formulation}
Let $J_i$ be contestant $i$'s judge score and $v_i$ the unknown fan vote share ($\sum_i v_i = 1$). We observe elimination $e$ and seek $\mathbf{v}$ consistent with the regime's rule.

\subsection{Percent Regime (S3--27)}
Define combined score $c_i = j_i + v_i$ where $j_i = J_i / \sum_k J_k$. The eliminated contestant has lowest $c_i$.

\paragraph{Likelihood.} We model elimination probability via softmax:
\begin{equation}
P(\text{elim}=e \mid \mathbf{v}) = \frac{\exp(-\tau c_e)}{\sum_i \exp(-\tau c_i)}, \quad \tau = 15.
\end{equation}

\paragraph{Optimization.} Maximize posterior (minimize negative log):
\begin{equation}
\min_{\mathbf{v}} \quad -\log P(e|\mathbf{v}) - \alpha \sum_i v_i \log v_i, \quad \text{s.t. } v_i \geq 0,\ \sum_i v_i = 1,
\end{equation}
where $\alpha = 0.05$ is entropy regularization (maximum entropy prior).

This is convex and solved via SLSQP, achieving \textbf{100\% constraint satisfaction}.

\subsection{Rank Regime (S1--2)}
Let $r^J_i$ and $r^F_i$ be judge and fan ranks (1 = best). Combined rank $c_i = r^J_i + r^F_i$. The largest $c_i$ is eliminated.

\paragraph{Construction.} Given eliminated contestant $e$:
\begin{enumerate}
    \item Non-eliminated contestants get fan ranks $1, 2, \ldots, n{-}1$ sorted by $r^J$.
    \item Set $r^F_e = n$ (worst rank).
    \item Verify: $c_e > c_p$ for all survivors $p$.
\end{enumerate}

\paragraph{Upset Detection.} When constraint fails (e.g., a contestant with good judge rank is eliminated), this indicates an \textbf{upset}---strong fan opposition overcame judges' preference. Our model identifies 4 such upsets in S1--2, consistent with historical records.

\paragraph{Rank to Share.} Convert ranks to shares:
\begin{equation}
v_i = \frac{\exp(-\lambda(r^F_i - 1))}{\sum_j \exp(-\lambda(r^F_j - 1))}, \quad \lambda = 0.5.
\end{equation}

\subsection{Bottom2 + Judges' Save (S28--34)}
Bottom two determined by combined rank; judges choose whom to eliminate.

\paragraph{MCMC Sampling.} Since the saved contestant is unknown, we sample possible bottom-2 pairs:
\begin{equation}
P(\text{elim } e \mid B_2 = \{e, b\}) = \frac{\exp(-\beta J_e)}{\exp(-\beta J_e) + \exp(-\beta J_b)}, \quad \beta = 1.
\end{equation}
This models judges' tendency to eliminate lower-scoring contestants.

\subsection{Finals Constraint}
In final weeks, top contestants are ranked (1st, 2nd, 3rd...). We enforce:
\begin{equation}
c_{\pi(1)} > c_{\pi(2)} > c_{\pi(3)} > \cdots
\end{equation}

%% ===========================================
\section{Certainty Quantification}
%% ===========================================

\subsection{Ensemble Perturbation}
The inverse problem may have multiple valid solutions. We quantify certainty via perturbation:

\begin{enumerate}
    \item Perturb judge scores: $J_i^{(k)} = J_i + \epsilon$, $\epsilon \sim N(0, 0.5^2)$.
    \item Re-solve for $K=20$ perturbations.
    \item Compute mean $\mu_i$ and std $\sigma_i$ across ensemble.
\end{enumerate}

\subsection{Certainty Metric}
\begin{equation}
\text{Certainty}_i = \frac{1}{1 + \text{CV}_i} = \frac{\mu_i}{\mu_i + \sigma_i},
\end{equation}
where $\text{CV} = \sigma/\mu$ is coefficient of variation.

\begin{itemize}
    \item Certainty $\to 1$: unique solution (high confidence).
    \item Certainty $\to 0$: many valid solutions (low confidence).
\end{itemize}

%% ===========================================
\section{Results}
%% ===========================================

\subsection{Constraint Satisfaction Rate (CSR)}

\begin{table}[h]
\centering
\caption{Model consistency by regime}
\begin{tabular}{lccc}
\toprule
\textbf{Regime} & \textbf{Seasons} & \textbf{CSR} & \textbf{Avg Certainty} \\
\midrule
Rank & 1--2 & 59.5\% & 0.913 \\
Percent & 3--27 & 100.0\% & 0.975 \\
Bottom2 & 28--34 & 77.3\% & 0.821 \\
\bottomrule
\end{tabular}
\label{tab:csr}
\end{table}

\paragraph{Why Rank CSR $<$ 100\%?} These are not model failures---they are \textbf{historical upsets} where fan votes strongly disagreed with judges. A contestant with good judge rank was eliminated because fans voted against them. This is valuable information: it reveals fan preferences that diverged from judge assessments.

\paragraph{Why Bottom2 CSR $<$ 100\%?} The judge save decision introduces additional uncertainty. When judges save a contestant with lower scores, our model (which assumes score-based preference) cannot fully explain the outcome.

\subsection{Certainty by Regime}

\begin{table}[h]
\centering
\caption{Certainty varies systematically by regime}
\begin{tabular}{lcc}
\toprule
\textbf{Regime} & \textbf{Avg Certainty} & \textbf{Interpretation} \\
\midrule
Percent & 0.975 & Very high---vote shares tightly constrained \\
Rank & 0.913 & High---but multiple rank permutations possible \\
Bottom2 & 0.821 & Moderate---judge choice adds randomness \\
\bottomrule
\end{tabular}
\end{table}

\subsection{Certainty by Week Type}
Certainty is \textbf{not uniform}:
\begin{itemize}
    \item \textbf{Elimination weeks}: Strong constraint $\Rightarrow$ high certainty.
    \item \textbf{No-elimination weeks}: No constraint $\Rightarrow$ lower certainty (prior-driven).
    \item \textbf{Finals}: Ranking constraint $\Rightarrow$ moderate certainty.
\end{itemize}

\subsection{Sample Estimates}
Table~\ref{tab:sample} shows fan vote estimates for Season 27 (Bobby Bones controversy).

\begin{table}[h]
\centering
\caption{Season 27 Week 10 estimates}
\begin{tabular}{lccc}
\toprule
\textbf{Celebrity} & \textbf{Judge Share} & \textbf{Fan Vote} & \textbf{Certainty} \\
\midrule
Bobby Bones & 0.248 & 0.271 & 0.982 \\
Milo Manheim & 0.256 & 0.245 & 0.978 \\
Evanna Lynch & 0.255 & 0.244 & 0.979 \\
Alexis Ren & 0.241 & 0.240 & 0.975 \\
\bottomrule
\end{tabular}
\label{tab:sample}
\end{table}

%% ===========================================
\section{Discussion}
%% ===========================================

\subsection{Is Certainty the Same for All?}
\textbf{No.} Certainty varies by:
\begin{enumerate}
    \item \textbf{Regime}: Percent $>$ Rank $>$ Bottom2.
    \item \textbf{Week type}: Elimination $>$ Finals $>$ No-elimination.
    \item \textbf{Contestant role}: Eliminated contestants have highest certainty (tightly constrained).
\end{enumerate}

\subsection{Interpreting Upsets}
When our model yields CSR $<$ 100\%, this identifies \textbf{upset eliminations} where:
\begin{itemize}
    \item A contestant with strong judge support was eliminated (fans disagreed).
    \item Or judges saved a contestant despite lower scores (subjective preference).
\end{itemize}
These cases are historically documented (e.g., early DWTS seasons had controversial eliminations).

\subsection{Model Limitations}
\begin{itemize}
    \item We estimate vote \emph{shares}, not absolute counts.
    \item Temporal smoothness prior assumes stable popularity---may miss sudden surges.
    \item Judge save behavior in Bottom2 is modeled probabilistically, not deterministically.
\end{itemize}

%% ===========================================
\section{Conclusion}
%% ===========================================
We developed a convex optimization framework for inverse estimation of DWTS fan votes, achieving:
\begin{itemize}
    \item 100\% constraint satisfaction for Percent regime (S3--27).
    \item Detection of historical upsets in Rank regime (S1--2).
    \item Certainty quantification via ensemble perturbation.
\end{itemize}
Key insight: certainty depends on regime, week type, and contestant role---it is \textbf{not uniform}.

\newpage
\begin{thebibliography}{99}
\bibitem{wiki} Wikipedia. Dancing with the Stars (American TV series). 
\bibitem{cvxpy} Diamond, S. and Boyd, S. CVXPY: A Python-Embedded Modeling Language for Convex Optimization. JMLR, 2016.
\end{thebibliography}

\end{document}
