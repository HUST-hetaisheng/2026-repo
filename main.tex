%%%%

%% MCM 2026 Problem C - DWTS Fan Vote Estimation%% MCM 2026 Problem C - DWTS Fan Vote Estimation

%%%%

\documentclass{mcmthesis}\documentclass{mcmthesis}

\mcmsetup{tstyle=\color{red}\bfseries,\mcmsetup{tstyle=\color{red}\bfseries,

        tcn = 2607256, problem = C,        tcn = 2607256, problem = C,

        sheet = true, titleinsheet = false, keywordsinsheet = true,        sheet = true, titleinsheet = false, keywordsinsheet = true,

        titlepage = false, abstract = true}        titlepage = false, abstract = true}



\usepackage{newtxtext,newtxmath}\usepackage{newtxtext,newtxmath}

\usepackage{indentfirst}\usepackage{indentfirst}

\usepackage{algorithm}\usepackage{algorithm}

\usepackage{algpseudocode}\usepackage{algpseudocode}

\usepackage{booktabs}\usepackage{booktabs}

\usepackage{float}

\usepackage{graphicx}\title{Inverse Estimation of Fan Votes in Dancing with the Stars}

\date{\today}

\title{Inverse Estimation of Fan Votes in Dancing with the Stars}

\date{\today}\begin{document}

\begin{abstract}

\begin{document}We address the inverse problem of estimating fan votes from DWTS elimination data across 34 seasons (2005--2024). The challenge: given only judges' scores and elimination outcomes, infer the hidden fan vote distribution. We develop regime-specific models for three distinct voting rules: Rank-based (S1--2), Percent-based (S3--27), and Bottom2+Judges' Save (S28--34). Our approach uses convex optimization with softmax likelihood and entropy regularization, achieving 100\% constraint satisfaction for Percent seasons and identifying ``upset'' eliminations in Rank seasons where strong fan opposition overcame judges' preferences. Certainty is quantified via ensemble perturbation, with Percent regime showing highest certainty (0.975), followed by Rank (0.913) and Bottom2 (0.821).

\begin{abstract}

We address the inverse problem of estimating fan votes from DWTS elimination data across 34 seasons (2005--2024). Given only judges' scores and elimination outcomes, we infer the hidden fan vote distribution. We develop regime-specific models for three distinct voting rules: Rank-based (S1--2), Percent-based (S3--27), and Bottom2+Judges' Save (S28--34). Our approach uses convex optimization with softmax likelihood and entropy regularization, achieving 100\% constraint satisfaction for Percent seasons. For Rank and Bottom2 regimes, constraint satisfaction rates below 100\% identify historical ``upset'' eliminations where fan votes strongly opposed judges' preferences. Certainty is quantified via ensemble perturbation, with Percent regime showing highest certainty (0.985), followed by Rank (0.905) and Bottom2 (0.804).\begin{keywords}

inverse problem; fan vote estimation; convex optimization; DWTS; certainty quantification

\begin{keywords}\end{keywords}

inverse problem; fan vote estimation; convex optimization; DWTS; certainty quantification\end{abstract}

\end{keywords}\maketitle

\end{abstract}\tableofcontents

\maketitle\newpage

\tableofcontents

\newpage%% ===========================================

\section{Introduction}
\subsection{Background}

DWTS (Dancing with the Stars) is a popular television competition where celebrity contestants are paired with professional dancers to compete in weekly dance performances. Elimination and advancement results are determined by combining judges' scores and audience votes. The judges assess technical proficiency, which can be subjective, while fan votes are influenced by factors such as celebrity popularity and charisma. As a result, the competition's outcomes have often sparked discussions and controversies, despite the show's attempts to improve the integration of these two elements. In recent years, with increasing audience concerns about fairness in competitive variety shows, DWTS faces a critical need for a fair, impartial, and effective scoring system. This system must ensure both the program's entertainment value and the confidentiality of fan votes, securing a balance between professionalism and popularity. This is essential for DWTS to maintain its viewership and grow its brand in future seasons.
% \begin{figure}[H]
%     \centering
%     \includegraphics[width=0.35\linewidth]{figures/dwtc背景图.png}
%     \caption{Dancing with the Stars poster}
%     \label{fig:placeholder}
% \end{figure}

\begin{figure}[H]
    \centering
    \begin{minipage}{0.37\linewidth}
        \centering
        \includegraphics[width=\linewidth]{figures/dwtc背景图.png}
    \end{minipage}
    \hspace{0.03\linewidth}
    \begin{minipage}{0.37\linewidth}
        \centering
        \includegraphics[width=\linewidth]{figures/dwtc问题综述图.png}
    \end{minipage}
    \textbf{\caption{Dancing with the Stars}}
    \label{fig:dwtc_overview}
\end{figure}



\subsection{Restatement of the Problem}

Given the background information and available data, this study addresses the following tasks.

\textbf{Task 1: Fan Vote Estimation.}
\textbf{1.1}\quad Construct a model to estimate weekly fan votes using judges’ scores, elimination outcomes, and contestant data.  
\textbf{1.2}\quad Evaluate the consistency of the estimates by testing whether they reproduce the observed weekly eliminations.  
\textbf{1.3}\quad Quantify the uncertainty of the estimated fan votes and examine whether it varies across contestants or weeks.

\textbf{Task 2: Comparison of Vote Combination and Elimination Rules.}
\textbf{2.1}\quad Compare the rank-based and percentage-based vote combination methods across seasons and analyze potential biases.  
\textbf{2.2}\quad Evaluate the impact of alternative elimination procedures, including judges selecting from the bottom two contestants.  
\textbf{2.3}\quad Recommend an appropriate combination and elimination approach for future seasons with justification.

\textbf{Task 3: Impact of Contestant and Partner Characteristics.}
\textbf{3.1}\quad Assess the influence of professional dancer traits and celebrity characteristics on overall performance and final results.  
\textbf{3.2}\quad Compare the effects of these factors on judges’ scores versus fan votes.

\textbf{Task 4: Alternative Scoring System.}
\textbf{4.1}\quad Design an alternative method for combining judges’ scores and fan votes.  
\textbf{4.2}\quad Demonstrate that the proposed system is more fair or otherwise improves the competition.


\subsection{Syntax (how to type \LaTeX\ commands --- these

\subsection{Three Voting Regimes}  are the rules)}

DWTS has used three distinct scoring rules over its history:

\begin{itemize}\lipsum[3]

    \item \textbf{Rank (S1--2)}: Combined rank = judge rank + fan rank. Highest sum eliminated.\begin{itemize}

    \item \textbf{Percent (S3--27)}: Combined score = judge share + fan share. Lowest sum eliminated.    \item \textbf{Rank} (S1--2): Combined rank = judge rank + fan rank. Highest sum eliminated.

    \item \textbf{Bottom2 (S28--34)}: Bottom two by combined rank; judges then choose whom to save.    \item \textbf{Percent} (S3--27): Combined score = judge share + fan share. Lowest sum eliminated.

\end{itemize}    \item \textbf{Bottom2} (S28--34): Bottom two by combined rank; judges then choose whom to save.

\end{itemize}

\begin{figure}[H]

\centering\begin{figure}[h]

\includegraphics[width=0.75\textwidth]{figures/rules_of_score_combining_and_couple_elimination.png}\centering

\caption{Scoring rule evolution across 34 seasons}\includegraphics[width=0.75\textwidth]{figures/rules_of_score_combining_and_couple_elimination.png}

\label{fig:rules}\caption{Scoring rule evolution across seasons}

\end{figure}\label{fig:rules}

\end{figure}

%% ===========================================

\section{Model: Inverse Estimation via Convex Optimization}%% ===========================================

%% ===========================================\section{Model: Inverse Estimation via Convex Optimization}

%% ===========================================

\subsection{Problem Formulation}

Let $J_i$ be contestant $i$'s judge score and $v_i$ the unknown fan vote share ($\sum_i v_i = 1$). We observe elimination $e$ and seek $\mathbf{v}$ consistent with the regime's rule. This is an \textbf{inverse problem}: given outputs (eliminations), infer inputs (votes).\subsection{Problem Formulation}

Let $J_i$ be contestant $i$'s judge score and $v_i$ the unknown fan vote share ($\sum_i v_i = 1$). We observe elimination $e$ and seek $\mathbf{v}$ consistent with the regime's rule.

\subsection{Percent Regime (S3--27)}

Define combined score $c_i = j_i + v_i$ where $j_i = J_i / \sum_k J_k$. The eliminated contestant has lowest $c_i$.\subsection{Percent Regime (S3--27)}

Define combined score $c_i = j_i + v_i$ where $j_i = J_i / \sum_k J_k$. The eliminated contestant has lowest $c_i$.

\paragraph{Likelihood.} We model elimination probability via softmax:

\begin{equation}\paragraph{Likelihood.} We model elimination probability via softmax:

P(\text{elim}=e \mid \mathbf{v}) = \frac{\exp(-\tau c_e)}{\sum_i \exp(-\tau c_i)}, \quad \tau = 15.\begin{equation}

\end{equation}P(\text{elim}=e \mid \mathbf{v}) = \frac{\exp(-\tau c_e)}{\sum_i \exp(-\tau c_i)}, \quad \tau = 15.

\end{equation}

\paragraph{Multi-elimination.} For weeks with $k$ eliminations, all eliminated contestants must be in the bottom $k$ by combined score:

\begin{equation}\paragraph{Optimization.} Maximize posterior (minimize negative log):

\log P(E_t \mid \mathbf{v}) = \sum_{e \in E_t} (-\tau c_e) - k \log \sum_i \exp(-\tau c_i).\begin{equation}

\end{equation}\min_{\mathbf{v}} \quad -\log P(e|\mathbf{v}) - \alpha \sum_i v_i \log v_i, \quad \text{s.t. } v_i \geq 0,\ \sum_i v_i = 1,

\end{equation}

\paragraph{Optimization.} Maximize posterior (minimize negative log):where $\alpha = 0.05$ is entropy regularization (maximum entropy prior).

\begin{equation}

\min_{\mathbf{v}} \quad -\log P(E|\mathbf{v}) - \alpha \sum_i v_i \log v_i, \quad \text{s.t. } v_i \geq 0,\ \sum_i v_i = 1,This is convex and solved via SLSQP, achieving \textbf{100\% constraint satisfaction}.

\end{equation}

where $\alpha = 0.05$ is entropy regularization (maximum entropy prior).\subsection{Rank Regime (S1--2)}

Let $r^J_i$ and $r^F_i$ be judge and fan ranks (1 = best). Combined rank $c_i = r^J_i + r^F_i$. The largest $c_i$ is eliminated.

This is convex and solved via SLSQP, achieving \textbf{100\% constraint satisfaction} across all 25 Percent seasons, including 33 multi-elimination weeks.

\paragraph{Construction.} Given eliminated contestant $e$:

\subsection{Rank Regime (S1--2)}\begin{enumerate}

Let $r^J_i$ and $r^F_i$ be judge and fan ranks (1 = best). Combined rank $c_i = r^J_i + r^F_i$. The largest $c_i$ is eliminated.    \item Non-eliminated contestants get fan ranks $1, 2, \ldots, n{-}1$ sorted by $r^J$.

    \item Set $r^F_e = n$ (worst rank).

\paragraph{Construction.} Given eliminated contestant $e$:    \item Verify: $c_e > c_p$ for all survivors $p$.

\begin{enumerate}\end{enumerate}

    \item Non-eliminated contestants get fan ranks $1, 2, \ldots, n{-}1$ sorted by $r^J$.

    \item Set $r^F_e = n$ (worst rank).\paragraph{Upset Detection.} When constraint fails (e.g., a contestant with good judge rank is eliminated), this indicates an \textbf{upset}---strong fan opposition overcame judges' preference. Our model identifies 4 such upsets in S1--2, consistent with historical records.

    \item Verify: $c_e > c_p$ for all survivors $p$.

\end{enumerate}\paragraph{Rank to Share.} Convert ranks to shares:

\begin{equation}

\paragraph{Upset Detection.} When constraint fails (a contestant with good judge rank is eliminated), this indicates an \textbf{upset}---strong fan opposition overcame judges' preference. Our model identifies such upsets in S1--2, consistent with historical records of controversial early-season eliminations.v_i = \frac{\exp(-\lambda(r^F_i - 1))}{\sum_j \exp(-\lambda(r^F_j - 1))}, \quad \lambda = 0.5.

\end{equation}

\paragraph{Rank to Share.} Convert ranks to vote shares:

\begin{equation}\subsection{Bottom2 + Judges' Save (S28--34)}

v_i = \frac{\exp(-\lambda(r^F_i - 1))}{\sum_j \exp(-\lambda(r^F_j - 1))}, \quad \lambda = 0.5.Bottom two determined by combined rank; judges choose whom to eliminate.

\end{equation}

\paragraph{MCMC Sampling.} Since the saved contestant is unknown, we sample possible bottom-2 pairs:

\subsection{Bottom2 + Judges' Save (S28--34)}\begin{equation}

Bottom two determined by combined rank; judges choose whom to eliminate.P(\text{elim } e \mid B_2 = \{e, b\}) = \frac{\exp(-\beta J_e)}{\exp(-\beta J_e) + \exp(-\beta J_b)}, \quad \beta = 1.

\end{equation}

\paragraph{Deterministic Construction.} We construct fan ranks such that the eliminated contestant is in the bottom two. The ``saved'' contestant (bottom-2 partner) is assumed to be the non-eliminated contestant with worst judge rank.This models judges' tendency to eliminate lower-scoring contestants.



\paragraph{Judge Choice Model.} Judges tend to eliminate lower-scoring contestants:\subsection{Finals Constraint}

\begin{equation}In final weeks, top contestants are ranked (1st, 2nd, 3rd...). We enforce:

P(\text{elim } e \mid B_2 = \{e, b\}) = \frac{\exp(-\beta J_e)}{\exp(-\beta J_e) + \exp(-\beta J_b)}, \quad \beta = 1.\begin{equation}

\end{equation}c_{\pi(1)} > c_{\pi(2)} > c_{\pi(3)} > \cdots

When judges save a lower-scoring contestant, our model cannot fully explain the outcome, resulting in CSR $<$ 100\%.\end{equation}



\subsection{Finals Constraint}%% ===========================================

In final weeks, top contestants are ranked (1st, 2nd, 3rd...). We enforce ordering with iterative correction:\section{Certainty Quantification}

\begin{equation}%% ===========================================

c_{\pi(1)} > c_{\pi(2)} > c_{\pi(3)} > \cdots

\end{equation}\subsection{Ensemble Perturbation}

This is applied after normalization to ensure constraint satisfaction.The inverse problem may have multiple valid solutions. We quantify certainty via perturbation:



%% ===========================================\begin{enumerate}

\section{Certainty Quantification}    \item Perturb judge scores: $J_i^{(k)} = J_i + \epsilon$, $\epsilon \sim N(0, 0.5^2)$.

%% ===========================================    \item Re-solve for $K=20$ perturbations.

    \item Compute mean $\mu_i$ and std $\sigma_i$ across ensemble.

\subsection{Ensemble Perturbation}\end{enumerate}

The inverse problem may have multiple valid solutions. We quantify certainty via perturbation:

\begin{enumerate}\subsection{Certainty Metric}

    \item Perturb judge scores: $J_i^{(k)} = J_i + \epsilon$, $\epsilon \sim N(0, 0.5^2)$, only for active contestants.\begin{equation}

    \item Re-solve for $K=20$ perturbations.\text{Certainty}_i = \frac{1}{1 + \text{CV}_i} = \frac{\mu_i}{\mu_i + \sigma_i},

    \item Compute mean $\mu_i$ and std $\sigma_i$ across ensemble.\end{equation}

\end{enumerate}where $\text{CV} = \sigma/\mu$ is coefficient of variation.



\subsection{Certainty Metric}\begin{itemize}

\begin{equation}    \item Certainty $\to 1$: unique solution (high confidence).

\text{Certainty}_i = \frac{1}{1 + \text{CV}_i} = \frac{\mu_i}{\mu_i + \sigma_i},    \item Certainty $\to 0$: many valid solutions (low confidence).

\end{equation}\end{itemize}

where $\text{CV} = \sigma/\mu$ is coefficient of variation.

\begin{itemize}%% ===========================================

    \item Certainty $\to 1$: unique solution (high confidence).\section{Results}

    \item Certainty $\to 0$: many valid solutions (low confidence).%% ===========================================

\end{itemize}

\subsection{Constraint Satisfaction Rate (CSR)}

%% ===========================================

\section{Results}\begin{table}[h]

%% ===========================================\centering

\caption{Model consistency by regime}

\subsection{Constraint Satisfaction Rate (CSR)}\begin{tabular}{lccc}

\toprule

\begin{table}[H]\textbf{Regime} & \textbf{Seasons} & \textbf{CSR} & \textbf{Avg Certainty} \\

\centering\midrule

\caption{Model consistency by regime}Rank & 1--2 & 59.5\% & 0.913 \\

\begin{tabular}{lccc}Percent & 3--27 & 100.0\% & 0.975 \\

\topruleBottom2 & 28--34 & 77.3\% & 0.821 \\

\textbf{Regime} & \textbf{Seasons} & \textbf{CSR} & \textbf{Avg Certainty} \\\bottomrule

\midrule\end{tabular}

Rank & 1--2 & 52.4\% & 0.905 \\\label{tab:csr}

Percent & 3--27 & 100.0\% & 0.985 \\\end{table}

Bottom2 & 28--34 & 79.7\% & 0.804 \\

\bottomrule\paragraph{Why Rank CSR $<$ 100\%?} These are not model failures---they are \textbf{historical upsets} where fan votes strongly disagreed with judges. A contestant with good judge rank was eliminated because fans voted against them. This is valuable information: it reveals fan preferences that diverged from judge assessments.

\end{tabular}

\label{tab:csr}\paragraph{Why Bottom2 CSR $<$ 100\%?} The judge save decision introduces additional uncertainty. When judges save a contestant with lower scores, our model (which assumes score-based preference) cannot fully explain the outcome.

\end{table}

\subsection{Certainty by Regime}

\paragraph{Why Rank CSR $<$ 100\%?} These are not model failures---they are \textbf{historical upsets} where fan votes strongly disagreed with judges. A contestant with good judge rank was eliminated because fans voted against them. This reveals fan preferences that diverged from judge assessments.

\begin{table}[h]

\paragraph{Why Bottom2 CSR $<$ 100\%?} The judge save decision introduces subjectivity. When judges save a contestant with lower scores (based on factors like entertainment value or personal preference), our score-based model cannot fully explain the outcome.\centering

\caption{Certainty varies systematically by regime}

\subsection{Certainty by Regime}\begin{tabular}{lcc}

\toprule

\begin{table}[H]\textbf{Regime} & \textbf{Avg Certainty} & \textbf{Interpretation} \\

\centering\midrule

\caption{Certainty varies systematically by regime}Percent & 0.975 & Very high---vote shares tightly constrained \\

\begin{tabular}{lcc}Rank & 0.913 & High---but multiple rank permutations possible \\

\topruleBottom2 & 0.821 & Moderate---judge choice adds randomness \\

\textbf{Regime} & \textbf{Avg Certainty} & \textbf{Interpretation} \\\bottomrule

\midrule\end{tabular}

Percent & 0.985 & Very high---vote shares tightly constrained \\\end{table}

Rank & 0.905 & High---but multiple rank permutations possible \\

Bottom2 & 0.804 & Moderate---judge choice adds randomness \\\subsection{Certainty by Week Type}

\bottomruleCertainty is \textbf{not uniform}:

\end{tabular}\begin{itemize}

\end{table}    \item \textbf{Elimination weeks}: Strong constraint $\Rightarrow$ high certainty.

    \item \textbf{No-elimination weeks}: No constraint $\Rightarrow$ lower certainty (prior-driven).

\subsection{Certainty by Week Type}    \item \textbf{Finals}: Ranking constraint $\Rightarrow$ moderate certainty.

Certainty is \textbf{not uniform}:\end{itemize}

\begin{itemize}

    \item \textbf{Elimination weeks}: Strong constraint $\Rightarrow$ high certainty.\subsection{Sample Estimates}

    \item \textbf{No-elimination weeks}: No constraint $\Rightarrow$ lower certainty (prior-driven).Table~\ref{tab:sample} shows fan vote estimates for Season 27 (Bobby Bones controversy).

    \item \textbf{Finals}: Ranking constraint $\Rightarrow$ moderate certainty.

    \item \textbf{Multi-elimination weeks}: All eliminated must be in bottom-$k$ $\Rightarrow$ high certainty.\begin{table}[h]

\end{itemize}\centering

\caption{Season 27 Week 10 estimates}

\subsection{Sample Estimates}\begin{tabular}{lccc}

Table~\ref{tab:sample} shows fan vote estimates for Season 27 (Bobby Bones controversy).\toprule

\textbf{Celebrity} & \textbf{Judge Share} & \textbf{Fan Vote} & \textbf{Certainty} \\

\begin{table}[H]\midrule

\centeringBobby Bones & 0.248 & 0.271 & 0.982 \\

\caption{Season 27 Week 10 estimates}Milo Manheim & 0.256 & 0.245 & 0.978 \\

\begin{tabular}{lccc}Evanna Lynch & 0.255 & 0.244 & 0.979 \\

\topruleAlexis Ren & 0.241 & 0.240 & 0.975 \\

\textbf{Celebrity} & \textbf{Judge Share} & \textbf{Fan Vote} & \textbf{Certainty} \\\bottomrule

\midrule\end{tabular}

Bobby Bones & 0.248 & 0.271 & 0.982 \\\label{tab:sample}

Milo Manheim & 0.256 & 0.245 & 0.978 \\\end{table}

Evanna Lynch & 0.255 & 0.244 & 0.979 \\

Alexis Ren & 0.241 & 0.240 & 0.975 \\%% ===========================================

\bottomrule\section{Discussion}

\end{tabular}%% ===========================================

\label{tab:sample}

\end{table}\subsection{Is Certainty the Same for All?}

\textbf{No.} Certainty varies by:

%% ===========================================\begin{enumerate}

\section{Discussion}    \item \textbf{Regime}: Percent $>$ Rank $>$ Bottom2.

%% ===========================================    \item \textbf{Week type}: Elimination $>$ Finals $>$ No-elimination.

    \item \textbf{Contestant role}: Eliminated contestants have highest certainty (tightly constrained).

\subsection{Is Certainty the Same for All?}\end{enumerate}

\textbf{No.} Certainty varies by:

\begin{enumerate}\subsection{Interpreting Upsets}

    \item \textbf{Regime}: Percent $>$ Rank $>$ Bottom2.When our model yields CSR $<$ 100\%, this identifies \textbf{upset eliminations} where:

    \item \textbf{Week type}: Elimination $>$ Finals $>$ No-elimination.\begin{itemize}

    \item \textbf{Contestant role}: Eliminated contestants have highest certainty (tightly constrained).    \item A contestant with strong judge support was eliminated (fans disagreed).

\end{enumerate}    \item Or judges saved a contestant despite lower scores (subjective preference).

\end{itemize}

\subsection{Interpreting Upsets}These cases are historically documented (e.g., early DWTS seasons had controversial eliminations).

When our model yields CSR $<$ 100\%, this identifies \textbf{upset eliminations} where:

\begin{itemize}\subsection{Model Limitations}

    \item A contestant with strong judge support was eliminated (fans disagreed).\begin{itemize}

    \item Or judges saved a contestant despite lower scores (subjective preference).    \item We estimate vote \emph{shares}, not absolute counts.

\end{itemize}    \item Temporal smoothness prior assumes stable popularity---may miss sudden surges.

These cases are historically documented in early DWTS seasons.    \item Judge save behavior in Bottom2 is modeled probabilistically, not deterministically.

\end{itemize}

\subsection{Model Limitations}

\begin{itemize}%% ===========================================

    \item We estimate vote \emph{shares}, not absolute counts.\section{Conclusion}

    \item No temporal smoothness constraint (weeks solved independently).%% ===========================================

    \item Judge save behavior in Bottom2 is modeled deterministically.We developed a convex optimization framework for inverse estimation of DWTS fan votes, achieving:

\end{itemize}\begin{itemize}

    \item 100\% constraint satisfaction for Percent regime (S3--27).

%% ===========================================    \item Detection of historical upsets in Rank regime (S1--2).

\section{Conclusion}    \item Certainty quantification via ensemble perturbation.

%% ===========================================\end{itemize}

We developed a convex optimization framework for inverse estimation of DWTS fan votes, achieving:Key insight: certainty depends on regime, week type, and contestant role---it is \textbf{not uniform}.

\begin{itemize}

    \item 100\% constraint satisfaction for Percent regime (S3--27), including 33 multi-elimination weeks.\newpage

    \item Detection of historical upsets in Rank (S1--2) and Bottom2 (S28--34) regimes.\begin{thebibliography}{99}

    \item Certainty quantification via ensemble perturbation.\bibitem{wiki} Wikipedia. Dancing with the Stars (American TV series). 

\end{itemize}\bibitem{cvxpy} Diamond, S. and Boyd, S. CVXPY: A Python-Embedded Modeling Language for Convex Optimization. JMLR, 2016.

\end{thebibliography}

\textbf{Key insight}: Certainty depends on regime, week type, and contestant role---it is \textbf{not uniform}. CSR $<$ 100\% is not a model failure but reveals genuine historical controversies.

\end{document}

\newpage
\begin{thebibliography}{99}
\bibitem{wiki} Wikipedia. Dancing with the Stars (American TV series). 
\bibitem{scipy} Virtanen, P. et al. SciPy 1.0: Fundamental Algorithms for Scientific Computing in Python. Nature Methods, 2020.
\end{thebibliography}

\end{document}
