%%
%% This is file `mcmthesis-demo.tex',
%% generated with the docstrip utility.
%%
%% The original source files were:
%%
%% mcmthesis.dtx  (with options: `demo')
%% !Mode:: "TeX:UTF-8"
%% -----------------------------------
%% This is a generated file.
%% 
%% Copyright (C) 2010 -- 2015 by latexstudio
%%       2014 -- 2019 by Liam Huang
%%       2019 -- present by latexstudio.net
%% 
%% License: The LaTeX Project Public License 1.3c
%% 
%% The Current Maintainer of this work is latexstudio.net.
%% 
\documentclass{mcmthesis}
 %\documentclass[CTeX = true]{mcmthesis}  % 当使用 CTeX 套装时请注释上一行使用该行的设置
\mcmsetup{tstyle=\color{red}\bfseries,%修改题号,队号的颜色和加粗显示,黑色可以修改为 black
        tcn = 2607256, problem = C, %修改队号,参赛题号
        sheet = true, titleinsheet = false, keywordsinsheet = true,%修改sheet显示信息
        titlepage = false, abstract = true}

  %四款字体可以选择
  %\usepackage{times}
  \usepackage{newtxtext,newtxmath} %CTeX 无此字体,可用 txfonts 替代,请使用新版 TeXLive.
  %\usepackage{palatino}
  %\usepackage{txfonts}

\usepackage{indentfirst}  %首行缩进,注释掉,首行就不再缩进。
\usepackage{lipsum}
\usepackage{algorithm}
\usepackage{algpseudocode}
\title{The \LaTeX{} Template for MCM Version \MCMversion}
\author{\small \href{https://www.latexstudio.net/}
  {\includegraphics[width=7cm]{mcmthesis-logo}}}
\date{\today}
\begin{document}
\begin{abstract}
\par Use this template to begin typing the first page (summary page) of your electronic report. This
template uses a 12-point Times New Roman font. Submit your paper as an Adobe PDF
electronic file (e.g. 1111111.pdf), typed in English, with a readable font of at least 12-point type. 

Do not include the name of your school, advisor, or team members on this or any page. 

Be sure to change the control number and problem choice above. 

You may delete these instructions as you begin to type your report here.  

\textbf{Follow us @COMAPMath on Twitter or COMAPCHINAOFFICIAL on Weibo for the
most up to date contest information.}

\begin{keywords}
keyword1; keyword2
\end{keywords}
\end{abstract}
\maketitle
%% Generate the Table of Contents, if it's needed.
\tableofcontents
\newpage
%%
%% Generate the Memorandum, if it's needed.
%% \memoto{\LaTeX{}studio}
%% \memofrom{Liam Huang}
%% \memosubject{Happy \TeX{}ing!}
%% \memodate{\today}
%% \memologo{\LARGE I'm pretending to be a LOGO!}
%% \begin{memo}[Memorandum]
%%   \lipsum[1-3]
%% \end{memo}
%%
\section{Introduction}
\subsection{Background}

DWTS (Dancing with the Stars) is a popular television competition where celebrity contestants are paired with professional dancers to compete in weekly dance performances. Elimination and advancement results are determined by combining judges' scores and audience votes. The judges assess technical proficiency, which can be subjective, while fan votes are influenced by factors such as celebrity popularity and charisma. As a result, the competition's outcomes have often sparked discussions and controversies, despite the show's attempts to improve the integration of these two elements. In recent years, with increasing audience concerns about fairness in competitive variety shows, DWTS faces a critical need for a fair, impartial, and effective scoring system. This system must ensure both the program's entertainment value and the confidentiality of fan votes, securing a balance between professionalism and popularity. This is essential for DWTS to maintain its viewership and grow its brand in future seasons.
% \begin{figure}[H]
%     \centering
%     \includegraphics[width=0.35\linewidth]{figures/dwtc背景图.png}
%     \caption{Dancing with the Stars poster}
%     \label{fig:placeholder}
% \end{figure}

\begin{figure}[H]
    \centering
    \begin{minipage}{0.37\linewidth}
        \centering
        \includegraphics[width=\linewidth]{figures/dwtc背景图.png}
    \end{minipage}
    \hspace{0.05\linewidth}
    \begin{minipage}{0.37\linewidth}
        \centering
        \includegraphics[width=\linewidth]{figures/dwtc问题综述图.png}
    \end{minipage}
    \textbf{\caption{Dancing with the Stars}}
    \label{fig:dwtc_overview}
\end{figure}



% \subsection{Restatement of the Problem}

% Given the background information and available data, this study addresses the following tasks.

% \textbf{Task 1: Fan Vote Estimation.}
% \textbf{1.1}\quad Construct a model to estimate weekly fan votes using judges’ scores, elimination outcomes, and contestant data.  
% \textbf{1.2}\quad Evaluate the consistency of the estimates by testing whether they reproduce the observed weekly eliminations.  
% \textbf{1.3}\quad Quantify the uncertainty of the estimated fan votes and examine whether it varies across contestants or weeks.

% \textbf{Task 2: Comparison of Vote Combination and Elimination Rules.}
% \textbf{2.1}\quad Compare the rank-based and percentage-based vote combination methods across seasons and analyze potential biases.  
% \textbf{2.2}\quad Evaluate the impact of alternative elimination procedures, including judges selecting from the bottom two contestants.  
% \textbf{2.3}\quad Recommend an appropriate combination and elimination approach for future seasons with justification.

% \textbf{Task 3: Impact of Contestant and Partner Characteristics.}
% \textbf{3.1}\quad Assess the influence of professional dancer traits and celebrity characteristics on overall performance and final results.  
% \textbf{3.2}\quad Compare the effects of these factors on judges’ scores versus fan votes.

% \textbf{Task 4: Alternative Scoring System.}
% \textbf{4.1}\quad Design an alternative method for combining judges’ scores and fan votes.  
% \textbf{4.2}\quad Demonstrate that the proposed system is more fair or otherwise improves the competition.
\subsection{Restatement of the Problem}

Given the background information and available data, this study addresses the following tasks.

\textbf{Task 1: Fan Vote Estimation.}\quad
A mathematical model is developed to estimate weekly fan vote totals for each contestant using judges’ scores, elimination outcomes, and contestant data. The model’s consistency is evaluated by its ability to reproduce observed weekly eliminations, and the uncertainty of the estimated fan votes is quantified and analyzed across contestants and weeks.

\textbf{Task 2: Comparison of Vote Combination and Elimination Rules.}\quad
Using the estimated fan votes, the rank-based and percentage-based methods for combining judges’ scores and fan votes are compared across seasons. The effects of alternative elimination procedures, including judges selecting from the bottom two contestants, are also evaluated, leading to a justified recommendation for future seasons.

\textbf{Task 3: Impact of Contestant and Partner Characteristics.}\quad
The impacts of professional dancer traits and celebrity characteristics on competition outcomes are analyzed, with a comparison of their influences on judges’ scores and fan votes.

\textbf{Task 4: Alternative Scoring System.}
An alternative scoring system combining \quad
judges’ scores and fan votes is proposed and evaluated to determine whether it improves fairness or other aspects of the competition.


\subsection{Syntax (how to type \LaTeX\ commands --- these
  are the rules)}

\lipsum[3]
\begin{itemize}
\item the angular velocity of the bat,
\item the velocity of the ball, and
\item the position of impact along the bat.
\end{itemize}
\lipsum[4]
\emph{center of percussion} [Brody 1986], \lipsum[5]

\begin{Theorem} \label{thm:latex}
\LaTeX
\end{Theorem}
\begin{Lemma} \label{thm:tex}
\TeX .
\end{Lemma}
\begin{proof}
The proof of theorem.
\end{proof}

\subsection{Other Assumptions}
\lipsum[6]
\begin{itemize}
\item
\item
\item
\item
\end{itemize}

\lipsum[7]

\section{Analysis of the Problem}
\begin{figure}[h]
\small
\centering
\includegraphics[width=8cm]{example-image-a}
\caption{The name of figure} \label{fig:aa}
\end{figure}

\lipsum[8] \eqref{aa}
\begin{equation}
a^2 \label{aa}
\end{equation}

\[
  \begin{pmatrix}{*{20}c}
  {a_{11} } & {a_{12} } & {a_{13} }  \\
  {a_{21} } & {a_{22} } & {a_{23} }  \\
  {a_{31} } & {a_{32} } & {a_{33} }  \\
  \end{pmatrix}
  = \frac{{Opposite}}{{Hypotenuse}}\cos ^{ - 1} \theta \arcsin \theta
\]
\lipsum[9]

\[
  p_{j}=\begin{cases} 0,&\text{if $j$ is odd}\\
  r!\,(-1)^{j/2},&\text{if $j$ is even}
  \end{cases}
\]

\lipsum[10]

\[
  \arcsin \theta  =
  \mathop{{\int\!\!\!\!\!\int\!\!\!\!\!\int}} \limits_\varphi
  {\mathop {\lim }\limits_{x \to \infty } \frac{{n!}}{{r!\left( {n - r}
  \right)!}}} \eqno (1)
\]
% ============================================
% Full modeling for Question 1 (all seasons)
% ============================================

\section{Model: Inverse Estimation of Fan Votes}

\subsection{Data Preparation and Notation}
Let $J_{i,t}$ be contestant $i$'s total judges score in week $t$ (sum of available judges; missing values ignored).
The data use $0$ as a placeholder after elimination, so the active set is
\begin{equation}
A_t=\{i: J_{i,t}>0\}.
\end{equation}
We treat the last positive score week as the primary elimination signal:
\begin{equation}
\mathrm{exit}(i)=\max\{t: J_{i,t}>0\}.
\end{equation}
The \texttt{results} text is used only to flag \texttt{Withdrew} cases, while finals ranking is taken from the native
\texttt{placement} column (complete for all contestants).

We estimate weekly fan vote \emph{shares}
\begin{equation}
v_{i,t}\ge 0,\qquad \sum_{i\in A_t} v_{i,t}=1,
\end{equation}
and define judge shares $j_{i,t}=J_{i,t}/\sum_{k\in A_t}J_{k,t}$. Let $E_t\subseteq A_t$ be the vote-determined
eliminated set (excluding \texttt{Withdrew}) and $S_t=A_t\setminus E_t$.

\subsection{Regime Overview}
The competition uses three different rules:
\begin{itemize}
\item \textbf{S1--2 (Rank)}: combined rank = judge rank + fan rank; the worst sum is eliminated.
\item \textbf{S3--27 (Percent)}: combined score = judge share + fan share; the smallest sum is eliminated.
\item \textbf{S28--34 (Bottom2 + judges' save)}: bottom two by combined rank; judges choose whom to eliminate.
\begin{figure}[htbp]
    \centering
    \includegraphics[width=0.5\linewidth]{figures/rules_of_score_combining_and_couple_elimination.png}
    \caption{Enter Caption}
    \label{fig:placeholder}
\end{figure}
\end{itemize}

\subsection{Regime A: Percent Seasons (S3--27)}
\paragraph{Goal.} Infer $v_{i,t}$ so that eliminated contestants are in the bottom-$|E_t|$ by combined score
$c_{i,t}=j_{i,t}+v_{i,t}$.

\paragraph{Hard consistency constraints.} For each elimination week,
\begin{equation}
c_{e,t}-c_{p,t}\le s_t,\qquad \forall e\in E_t,\ \forall p\in S_t,
\label{eq:percent_hard}
\end{equation}
with slack $s_t\ge 0$ to absorb ties and special episodes. If $E_t=\varnothing$, no ordering constraint is imposed.

\paragraph{Two-stage convex optimization.} Consistency is prioritized via a two-stage solve:
\begin{enumerate}
\item \textbf{Stage 1 (feasibility):} minimize $s_t$ subject to Eq.~\eqref{eq:percent_hard} and the simplex constraint.
\item \textbf{Stage 2 (tie-break):} restrict $s_t\le s_t^\star+\varepsilon$ and minimize
\begin{equation}
\beta\,\mathcal{L}(\bm v_t)-\alpha\sum_{i\in A_t}v_{i,t}\log v_{i,t},
\end{equation}
where $\mathcal{L}$ is a softmax negative log-likelihood for multi-elimination weeks:
\begin{equation}
\mathcal{L}(\bm v_t)=\tau\sum_{e\in E_t}c_{e,t}+|E_t|\,\log\!\left(\sum_{i\in A_t}e^{-\tau c_{i,t}}\right).
\end{equation}
\end{enumerate}
This preserves the elimination rule while selecting a neutral, high-entropy solution within the feasible set.

\paragraph{Finals and no-elimination weeks.} For the final week, we enforce the ranking order
$c_{\pi(1),T}>c_{\pi(2),T}>\cdots$ using \texttt{placement}. For no-elimination weeks,
we interpolate $v_{i,t}$ from adjacent weeks with the same active set, avoiding unrealistic uniform or
judge-equals-fan assumptions.

\subsection{Regime B: Rank Seasons (S1--2)}
Let $r^J_{i,t}$ be the judge rank (1=best). Fan ranks $r^F_{i,t}$ must form a \emph{permutation} of
$\{1,\ldots,|A_t|\}$. The combined rank is
\begin{equation}
c_{i,t}=r^J_{i,t}+r^F_{i,t},
\end{equation}
and eliminated contestants are those with the largest $c_{i,t}$.
We construct a feasible permutation by assigning the worst fan ranks to eliminated contestants and
the best fan ranks to contestants with weak judge ranks, then perform swaps to satisfy the elimination constraint.
If no permutation makes all eliminated contestants the worst, we label the week as an \emph{upset}.
Fan ranks are mapped to vote shares by
\begin{equation}
v_{i,t}=\frac{\exp(-\lambda(r^F_{i,t}-1))}{\sum_{p\in A_t}\exp(-\lambda(r^F_{p,t}-1))}.
\end{equation}

\subsection{Regime C: Bottom2 + Judges' Save (S28--34)}
The bottom two are the contestants with the two highest combined ranks.
In single-elimination weeks, the partner $b$ in the bottom two is unobserved.
We enumerate each $b\in A_t\setminus\{e\}$, assign fan ranks to make $\{e,b\}$ the bottom two
while keeping the remaining ranks aligned with judges, and choose the feasible candidate with minimal deviation.
If no candidate yields a feasible bottom-two pair, we relax the constraint and record the week accordingly.
Multi-elimination weeks in this regime fall back to the rank-permutation construction.

\subsection{Validation and Certainty}
\paragraph{Consistency.} We compute a \textbf{constraint satisfaction rate} (CSR): the fraction of elimination weeks
whose inferred votes reproduce the observed elimination rule.

\paragraph{Certainty.} We perturb only active judge scores ($J_{i,t}>0$), re-solve the model, and compute
\begin{equation}
\text{Certainty}(i,t)=\frac{1}{1+\text{CV}(i,t)}=\frac{\mu_{i,t}}{\mu_{i,t}+\sigma_{i,t}},
\end{equation}
where $\mu_{i,t}$ and $\sigma_{i,t}$ are the ensemble mean and standard deviation of $v_{i,t}$. The reported
results below use $K=5$ ensemble runs for tractable runtime.

\begin{algorithm}[H]
\caption{Inverse Estimation Workflow (All Seasons)}
\begin{algorithmic}[1]
\State \textbf{Input:} Judge scores $J_{i,t}$, results text, placements.
\State \textbf{Output:} Fan vote shares $\hat v_{i,t}$, CSR, certainty statistics.
\State Build $A_t=\{i:J_{i,t}>0\}$ and classify weeks by $E_t$.
\For{each season}
    \If{Percent (S3--27)}
        \State Stage 1: minimize slack $s_t$ under Eq.~\eqref{eq:percent_hard}.
        \State Stage 2: enforce $s_t\le s_t^\star+\varepsilon$ and minimize
        $\beta\,\mathcal{L}(\bm v_t)-\alpha\sum_i v_{i,t}\log v_{i,t}$.
    \ElsIf{Rank (S1--2)}
        \State Construct a valid fan-rank permutation; swap ranks until eliminations are worst.
    \ElsIf{Bottom2 (S28--34)}
        \State Enumerate partner $b$ so that $\{e,b\}$ is bottom two; choose minimal-deviation solution.
    \EndIf
    \State Enforce finals ordering from \texttt{placement}; interpolate $v_{i,t}$ for no-elimination weeks.
    \State Ensemble: perturb $J_{i,t}$ (only if $J_{i,t}>0$), re-solve, compute $\mu,\sigma,\text{Certainty}$.
    \State Validate eliminations and compute CSR.
\EndFor
\end{algorithmic}
\end{algorithm}

\section{Model Results and Analysis}

\subsection{Consistency with Eliminations}
Table~\ref{tab:csr} shows the CSR by regime. Percent seasons achieve 100\% consistency by construction
of hard constraints. Bottom2 seasons remain below 100\% because the bottom-two partner is unobserved
and some weeks require relaxed constraints.

\begin{table}[H]
\centering
\caption{Constraint Satisfaction Rate by Regime}
\begin{tabular}{lcc}
\toprule
\textbf{Regime} & \textbf{Seasons} & \textbf{CSR} \\
\midrule
Rank & 1--2 & 100.0\% \\
Percent & 3--27 & 100.0\% \\
Bottom2 & 28--34 & 79.7\% \\
\bottomrule
\end{tabular}
\label{tab:csr}
\end{table}

\subsection{Certainty of Estimated Fan Votes}
Average certainty by regime is summarized in Table~\ref{tab:certainty}. Percent seasons are most constrained,
while Bottom2 seasons are least certain because the judges' save introduces additional ambiguity.

\begin{table}[H]
\centering
\caption{Average Certainty by Regime (K=5 Ensemble)}
\begin{tabular}{lcc}
\toprule
\textbf{Regime} & \textbf{Avg Certainty} & \textbf{Interpretation} \\
\midrule
Rank & 0.965 & High (permutation ambiguity) \\
Percent & 0.984 & Very high (hard constraints) \\
Bottom2 & 0.835 & Moderate (unknown partner + judge save) \\
\bottomrule
\end{tabular}
\label{tab:certainty}
\end{table}

\subsection{Certainty Variation by Week Type}
Certainty is not uniform across week types. Table~\ref{tab:weektype_certainty} reports the mean certainty
for contestant-week observations by week type.

\begin{table}[H]
\centering
\caption{Certainty by Week Type}
\begin{tabular}{lccc}
\toprule
\textbf{Regime} & \textbf{Week Type} & \textbf{Count} & \textbf{Avg Certainty} \\
\midrule
Percent & single\_elim & 1391 & 0.986 \\
Percent & no\_elim\_interp & 318 & 0.988 \\
Percent & multi\_elim\_2 & 200 & 0.969 \\
Percent & multi\_elim\_3 & 6 & 0.901 \\
Percent & finals & 82 & 1.000 \\
Bottom2 & single\_elim & 370 & 0.824 \\
Bottom2 & bottom2\_relaxed & 108 & 0.818 \\
Bottom2 & no\_elim\_interp & 108 & 0.826 \\
Bottom2 & multi\_elim\_2 & 85 & 0.851 \\
Bottom2 & finals & 31 & 1.000 \\
\bottomrule
\end{tabular}
\label{tab:weektype_certainty}
\end{table}

\paragraph{Interpretation.}
Multi-elimination weeks are the most uncertain due to tighter ordering requirements, while finals weeks are most
stable because the ranking constraint is strong and the active set is small.

\section{A Summary}
\lipsum[6]

\section{Evaluate of the Mode}

\section{Strengths and weaknesses}
\lipsum[12]

\subsection{Strengths}
\begin{itemize}
\item \textbf{Applies widely}\\
This  system can be used for many types of airplanes, and it also
solves the interference during  the procedure of the boarding
airplane,as described above we can get to the  optimization
boarding time.We also know that all the service is automate.
\item \textbf{Improve the quality of the airport service}\\
Balancing the cost of the cost and the benefit, it will bring in
more convenient  for airport and passengers.It also saves many
human resources for the airline.
\end{itemize}

\subsection{How to cite?}
bibliography cite use \cite{1,2,3}

AI cite use \AIcite{AI1,AI2,AI3}

\begin{thebibliography}{99}
\bibitem{1} D.~E. KNUTH   The \TeX{}book  the American
Mathematical Society and Addison-Wesley
Publishing Company , 1984-1986.
\bibitem{2}Lamport, Leslie,  \LaTeX{}: `` A Document Preparation System '',
Addison-Wesley Publishing Company, 1986.
\bibitem{3}\url{https://www.latexstudio.net/}
\end{thebibliography}

\begin{appendices}

\section{First appendix}

In addition, your report must include a letter to the Chief Financial Officer (CFO) of the Goodgrant Foundation, Mr. Alpha Chiang, that describes the optimal investment strategy, your modeling approach and major results, and a brief discussion of your proposed concept of a return-on-investment (ROI). This letter should be no more than two pages in length.

\begin{letter}{Dear, Mr. Alpha Chiang}

\lipsum[1-2]

\vspace{\parskip}

Sincerely yours,

Your friends

\end{letter}
Here are simulation programmes we used in our model as follow.\\

\textbf{\textcolor[rgb]{0.98,0.00,0.00}{Input matlab source:}}
\lstinputlisting[language=Matlab]{./code/mcmthesis-matlab1.m}

\section{Second appendix}

some more text \textcolor[rgb]{0.98,0.00,0.00}{\textbf{Input C++ source:}}
\lstinputlisting[language=C++]{./code/mcmthesis-sudoku.cpp}

\end{appendices}


\AImatter

\begin{ReportAiUse}{9}
\bibitem{AI1}
Bing AI\\
Query1: write a sort algorithm\\
Output: Sure, I can help you with that. Here is a simple implementation of the selection sort algorithm in Python:
\begin{lstlisting}[language=python]
def selection_sort(arr):
    for i in range(len(arr)):
        min_idx = i
        for j in range(i + 1, len(arr)):
            if arr[min_idx] > arr[j]:
                min_idx = j
        arr[i], arr[min_idx] = arr[min_idx], arr[i]
    return arr
\end{lstlisting}
This algorithm sorts an array by repeatedly finding the minimum element from the unsorted part of the array and putting it at the beginning. \href{https://www.geeksforgeeks.org/sorting-algorithms/}{The time complexity of this algorithm is O($n^2$)}.

There are many other sorting algorithms available such as bubble sort, insertion sort, merge sort, quick sort, heap sort, and more. Each of these algorithms has its own advantages and disadvantages. You can learn more about them on GeeksforGeeks and Programiz.

I hope this helps!

\bibitem{AI2}
OpenAl Enie (Nov 5, 2023 version, Ernie 4.0)\\
Query1: <insert the exact wording of any subsequent input into the Al tool>\\
Output: <insert the complete output from the second query>

\bibitem{AI3}
Github CoPilot (Feb 3, 2024 version)\\
Query1: <insert the exact wording you input into the Al tool>\\
Output: <insert the complete output fron the Al tool>

\bibitem{AI4}
Google Bard (Feb 2,2024 version)\\
Query: <insert the exact wording of your query>\\
Output: <irsert the complete output from the Al tool>

\bibitem{AI5}
Bing AI\\
Query: write a FFT algorithm\\
Output: Sure, here is a Python implementation of the Fast Fourier Transform (FFT) algorithm:
\begin{lstlisting}[language=python]
import numpy as np

def fft(x):
    n = len(x)
    if n == 1:
        return x
    even = fft(x[0::2])
    odd = fft(x[1::2])
    T = [np.exp(-2j * np.pi * k / n) * odd[k] for k in range(n // 2)]
    return [even[k] + T[k] for k in range(n // 2)] + [even[k] - T[k] for k in range(n // 2)]
\end{lstlisting}
The FFT is a widely used algorithm for computing the \textbf{Discrete Fourier Transform (DFT)} of a sequence. It reduces the number of computations needed from O(n$^2$) to O(n log n) 12. The basic idea of the FFT is to apply divide and conquer. We divide the coefficient vector of the polynomial into two vectors, recursively compute the DFT for each of them, and combine the results to compute the DFT of the complete polynomial 3.

I hope this helps!

\end{ReportAiUse}

\end{document}
%% 
%% This work consists of these files mcmthesis.dtx,
%%                                   figures/ and
%%                                   code/,
%% and the derived files             mcmthesis.cls,
%%                                   mcmthesis-demo.tex,
%%                                   README,
%%                                   LICENSE,
%%                                   mcmthesis.pdf and
%%                                   mcmthesis-demo.pdf.
%%
%% End of file `mcmthesis-demo.tex'.
