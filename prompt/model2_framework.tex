% =================================================================================
% MODEL 2: Comparative Analysis of Rank vs. Percentage Methods
% =================================================================================

\section{Model 2: Rank vs. Percentage Comparison}

% ---------------------------------------------------------------------------------
\subsection{Problem Decomposition}
% ---------------------------------------------------------------------------------

We decompose Question 2 into five sub-problems:

\begin{enumerate}
    \item[\textbf{Part A.}] \textbf{Cross-Season Application}: Apply both Rank and Percentage methods to all 34 seasons.
    
    \item[\textbf{Part B.}] \textbf{Method Agreement Analysis}: Quantify agreement/disagreement patterns.
    
    \item[\textbf{Part C.}] \textbf{Fan Vote Sensitivity}: Determine which method favors fan votes more.
    
    \item[\textbf{Part D.}] \textbf{Controversial Case Studies}: Analyze 4 specific contestants under counterfactual scenarios.
    
    \item[\textbf{Part E.}] \textbf{Bottom-2 Forward Validation}: Handle probabilistic elimination via Monte Carlo simulation.
\end{enumerate}

% ---------------------------------------------------------------------------------
\subsection{Part A: Cross-Season Method Application}
% ---------------------------------------------------------------------------------

For each week $w$ with $N$ contestants:

\textbf{Rank Method:}
\begin{equation}
    R_{i,w} = r^J_{i,w} + r^F_{i,w}, \quad e^{\text{Rank}}_w = \arg\max_i R_{i,w}
\end{equation}

\textbf{Percentage Method:}
\begin{equation}
    S_{i,w} = j_{i,w} + f_{i,w}, \quad e^{\text{Pct}}_w = \arg\min_i S_{i,w}
\end{equation}

% ---------------------------------------------------------------------------------
\subsection{Part B: Agreement Analysis}
% ---------------------------------------------------------------------------------

\begin{align}
    \text{Agreement Rate} &= \frac{\#\{w : e^{\text{Rank}}_w = e^{\text{Pct}}_w\}}{\text{Total weeks}}
\end{align}

% ---------------------------------------------------------------------------------
\subsection{Part C: Fan Vote Sensitivity Index}
% ---------------------------------------------------------------------------------

When methods disagree:
\begin{equation}
    \Delta_{\text{fan}} = \hat{f}_{e^{\text{Rank}}} - \hat{f}_{e^{\text{Pct}}}
\end{equation}

If $\Delta_{\text{fan}} > 0$: Rank eliminates more popular $\Rightarrow$ Percentage is fan-friendly.

% ---------------------------------------------------------------------------------
\subsection{Part D: Controversial Case Studies}
% ---------------------------------------------------------------------------------

\begin{table}[H]
\centering
\begin{tabular}{llcc}
\toprule
\textbf{Contestant} & \textbf{Season} & \textbf{Bottom Judge Weeks} & \textbf{Placement} \\
\midrule
Jerry Rice & 2 (Rank) & 5 & 2nd \\
Billy Ray Cyrus & 4 (Pct) & 6 & 5th \\
Bristol Palin & 11 (Pct) & 12 & 3rd \\
Bobby Bones & 27 (Pct) & [X] & Winner \\
\bottomrule
\end{tabular}
\end{table}

% ---------------------------------------------------------------------------------
\subsection{Part E: Bottom-2 Forward Validation}
% ---------------------------------------------------------------------------------

For Season 28--34, elimination is \textbf{stochastic}. We cannot deterministically predict who gets eliminated—only compute probabilities. Validation requires Monte Carlo simulation.

\subsubsection{E.1 The Challenge}

Unlike Rank/Percentage methods where $e_w = \arg\min/\max(\cdot)$ gives a deterministic answer, Bottom-2 mechanism has:
\begin{equation}
    P(\text{Elim } a \mid \mathcal{B}_w = \{a, b\}) = \sigma(\beta(J_b - J_a))
\end{equation}

This yields a \textit{probability}, not a deterministic outcome. How do we validate?

\subsubsection{E.2 Step 1: Identify Bottom-2}

From combined ranks:
\begin{equation}
    \mathcal{B}_w = \{a, b\} \quad \text{where } R_{a,w}, R_{b,w} \text{ are the two highest (worst)}
\end{equation}

\subsubsection{E.3 Step 2: Compute Elimination Probability}

For $\{a, b\} \in \mathcal{B}_w$:
\begin{equation}
    P_a = P(\text{Elim } a) = \frac{1}{1 + \exp(\beta(J_a - J_b))}
\end{equation}

\subsubsection{E.4 Step 3: Monte Carlo Season Simulation}

\begin{algorithm}[H]
\caption{Bottom-2 Forward Simulation}
\begin{algorithmic}[1]
\Require Contestants $\mathcal{C}_1$, fan votes $\hat{\mathbf{f}}$, judge scores $\mathbf{J}$, $M = 10000$
\Ensure Placement distributions

\For{$m = 1$ to $M$}
    \State $\mathcal{C} \gets \mathcal{C}_1$ \Comment{Reset}
    \For{$w = 1$ to final week}
        \State Compute $\mathcal{B}_w = \{a, b\}$
        \State Sample $u \sim \text{Uniform}(0,1)$
        \If{$u < P_a$}
            \State Eliminate $a$
        \Else
            \State Eliminate $b$
        \EndIf
        \State Update $\mathcal{C}$
    \EndFor
    \State Record placements for run $m$
\EndFor
\end{algorithmic}
\end{algorithm}

\subsubsection{E.5 Validation Metrics}

\textbf{Metric 1: Weekly Probability Calibration}

Did the actually eliminated contestant have the higher elimination probability?
\begin{equation}
    \text{Calibration} = \frac{1}{|\mathcal{W}|} \sum_w \mathbf{1}\left[P(\text{Elim } e_w^{\text{actual}}) \geq 0.5\right]
\end{equation}

Expected: $\sim 80\%$ if model is well-calibrated.

\textbf{Metric 2: Placement Distribution Match}

Compare simulated placement distribution to actual:
\begin{equation}
    \text{Placement}_i^{\text{sim}} \sim \text{Empirical from } M \text{ runs}
\end{equation}

Check if actual placement falls within 90\% CI of simulated distribution.

\textbf{Metric 3: Survival Probability for Controversial Contestants}

\begin{equation}
    P_i^{\text{survive to week } W} = \prod_{w=1}^{W-1} P(\text{not eliminated in week } w)
\end{equation}

For Bristol Palin (12 bottom-judge weeks), if $P^{\text{survive}} < 0.01$ but she reached 3rd place $\Rightarrow$ extreme fan vote support confirmed.

\subsubsection{E.6 Interpretation}

\begin{itemize}
    \item \textbf{High $P^{\text{survive}}$, survived}: Model consistent with outcome
    \item \textbf{Low $P^{\text{survive}}$, survived}: Fan votes exceptionally high (overcame odds)
    \item \textbf{High $P^{\text{survive}}$, eliminated}: Model overestimated fan support
    \item \textbf{Low $P^{\text{survive}}$, eliminated}: Expected outcome
\end{itemize}

\subsubsection{E.7 Counterfactual: What if Bottom-2 used in earlier seasons?}

Apply Bottom-2 + Judges' Save to Seasons 3--27 (originally Percentage):
\begin{enumerate}
    \item Compute Bottom-2 each week using fan vote estimates
    \item Simulate judges' save with probability model
    \item Compare expected placements to actual
\end{enumerate}

This reveals whether Bottom-2 would have changed outcomes for controversial contestants.
